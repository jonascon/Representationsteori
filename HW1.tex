\documentclass[11pt]{amsart}
\usepackage[marginratio=1:1, headheight=27pt,
footskip = 13pt, a4paper, total={6.5in, 8in}]{geometry}                % See geometry.pdf to learn the layout options. There are lots.
\geometry{letterpaper}                   % ... or a4paper or a5paper or ...
%\geometry{landscape}                % Activate for for rotated page geometry
%\usepackage[parfill]{parskip}    % Activate to begin paragraphs with an empty line rather than an indent
\usepackage{graphicx}
\usepackage[font=small,labelfont=bf]{caption} % Required for specifying captions to tables
\usepackage{dirtytalk}
%\usepackage[version=3]{mhchem}
%\usepackage{subcaption}
\usepackage{subfig}
%\usepackage{epstopdf}
\usepackage{booktabs}
\usepackage[compact,small]{titlesec}
\usepackage{complexity}
%\usepackage[sc, osf]{mathpazo} % add possibly `sc` and `osf` options
%\usepackage{eulervm}
%\usepackage{textcomp}

\clubpenalty = 10000
\widowpenalty = 10000
%\usepackage[altbullet]{lucidabr}
\usepackage{float}
\usepackage{siunitx}
\usepackage[justification=centering]{caption}
\usepackage[
colorlinks = true,
urlcolor = blue
]{hyperref}
\usepackage{mathrsfs}
\usepackage{amssymb}
\usepackage{mathtools}
\usepackage{amsmath}
\usepackage{amsthm}
\renewcommand\qedsymbol{$\blacksquare$}

\usepackage[section]{placeins}
\DeclareGraphicsRule{.tif}{png}{.png}{`convert #1 `dirname #1`/`basename #1 .tif`.png}

%\usepackage[swedish]{babel}
\usepackage[T1]{fontenc}
\usepackage[utf8]{inputenc}

\usepackage{comment}
\usepackage{enumitem}
\titleformat{\section}[block]{\Large\scshape\centering}{\thesection.}{1em}{}
\titleformat{\subsection}[block]{\Large}{\thesubsection.}{1em}{}

\usepackage{pgfplots}
\pgfplotsset{compat = 1.15}

\usepackage[T1]{fontenc}

\usepackage{listings}
\lstset{basicstyle=\ttfamily,breaklines=true}
\usepackage{color} %red, green, blue, yellow, cyan, magenta, black, white
\definecolor{mygreen}{RGB}{28,172,0} % color values Red, Green, Blue
\definecolor{mylilas}{RGB}{170,55,241}

%\usepackage{times}
%\usepackage{kpfonts}
%\usepackage{txfonts}
%\usepackage{newtx}
%\usepackage{stix}
%\usepackage[osf,proportional]{libertine}
%\usepackage{lmodern}
\usepackage[charter]{mathdesign}

\usepackage{microtype}
%\usepackage{stix}
\usepackage[compact,small]{titlesec}
\titleformat*{\section}{\Large\bfseries\scshape}

\titleformat*{\subsection}{\large\bfseries\scshape}
\titleformat*{\subsubsection}{\large\bfseries\scshape}
%\renewcommand{\section}{\section*}
\newcommand{\ibits}{\{0, 1\}^*}
%\renewcommand{\familydefault}{\sfdefault}
\usepackage{marginnote}
\renewcommand*{\marginfont}{\color{red}\sffamily\footnotesize}
\usepackage{hyperref, xcolor}

%\usepackage{makeidx}
\definecolor{winered}{rgb}{0.5,0,0}
\hypersetup{
     pdfauthor={JAG},
     pdfsubject={Hyperlinks in LaTeX},
     pdftitle={main.tex},
     pdfkeywords={LaTeX, PDF, hyperlinks}
%    colorlinks=false,
     pdfborder={0 0 0},
%You can set individual colors for links as below:
colorlinks=true,
  linkcolor=winered,
urlcolor={winered},
filecolor={winered},
citecolor={winered},
allcolors={winered}
}

\usepackage[english]{babel} 
\usepackage[
backend=biber,
style=numeric,
hyperref=true,
%natbib
]{biblatex}
\DeclareLanguageMapping{swedish}{swedish-apa}
\addbibresource{Komplexitetsteori.bib}

\usepackage[T1]{fontenc}

%\usepackage{fouriernc}

\usepackage{thmtools}
\usepackage{fancyhdr}
\pagestyle{fancy}
\usepackage{csquotes}
\newsavebox{\myheadbox}
\fancyhf{}
%\begin{flushright}
\lhead{
Jonas Conneryd \\ \url{conneryd@kth.se}}
%\end{flushright}
\rhead{970731-7559  \\
\the\year
}
\chead{\Large{\textbf{\textsf{Homework 1} \\ \vspace{-4pt}\textsf{\normalsize MM8021 Representation Theory for Finite Groups}}}}
\cfoot{\thepage}
\declaretheoremstyle[headfont=\bfseries\sffamily]{normalhead}
\interfootnotelinepenalty=10000
%\title{\vspace{-2cm}\textbf{\textsf{ Homework 3}}}
\date{}
%\author{Jonas Conneryd \\
%conneryd@kth.se \\ 970731-7559}

\renewcommand{\C}{\mathbb{C}}
\newcommand{\GL}{\mathsf{GL}}
\newcommand{\Z}{\mathbb{Z}}
\renewcommand{\R}{\mathbb{R}}
\newcommand{\Res}{\mathsf{Res}^{D_{2n}}_{\langle x\rangle}}
\newcommand{\triv}{\mathbb{1}}
\newcommand{\Char}{\mathsf{Char}}
\newcommand{\Fix}{\mathsf{Fix}}
\newcommand{\Span}{\mathsf{Span}}


\begin{document}
%\maketitle
\thispagestyle{fancy}
\theoremstyle{normalhead}
\newtheorem{problem}{Problem}
\newtheorem{lemma}{Lemma}

\begin{problem}
Let the symmetric group $S_n$ act on the vector space $k^n$ in the usual way: $S_n$ acts on the standard coordinate basis $(e_1, \ldots, e_n)$ by $\sigma e_i=e_{\sigma(i)}$ for $\sigma \in S_n$ and we extend this action to $k^n$ by requiring it to be linear : $\sigma\left(\sum c_i e_i\right) \coloneqq \sum c_i \sigma(e_i)$. 
\begin{enumerate}[font=\normalfont,label=\textbf{(\alph*)}]

\item Let $a\coloneqq (a_1, \ldots, a_n) \in k^n$. Show that $\sigma(a) = (a_{\sigma^{-1}(1)}, \ldots , a_{\sigma^{-1}(n)})$. [So care must be taken when we say \say{$S_n$ acts by permuting the coordinates}.]

\item Show that the resulting representation $\rho: S_n \to \mathsf{GL}(n, k)$ contains a subrepresentation isomorphic to the trivial representation $\mathbb{1}$. 
\item If $k$ has characteristic zero, show that $\rho = \mathbb{1} \oplus \rho'$ is the direct sum of the trivial representation and a representation $\rho'$ of dimension $n-1$. 

\item Assume $k$ has characteristic 2 and $n = 2$. Show that $\rho$ is not a direct sum of two nontrivial subspaces. In other words, $\rho$ is reducible but not completely reducible. What goes wrong as opposed to \textbf{(c)}?

\item Let $p$ be a prime and assume $k$ has characteristic $p$. Show that $\rho$ is a direct sum as in \textbf{(c)} if and only if $p$ does not divide $n$. 

\end{enumerate}
\end{problem}

\begin{proof}[Solution]
\hfill
\begin{enumerate}[font=\normalfont,label=\textbf{(\alph*)}, wide]

\item We have 
\[
\sigma(a) = \sigma\left( \sum_i a_ie_i \right) = \sum_i a_i e_{\sigma(i)}.
\]
Since a permutation is a bijection, if we change variable to $j = \sigma^{-1}(i)$, we are simply changing the order of summation. This yields the same summand in a finite sum. Therefore
\[
\sigma(a) = \sum_i a_i e_{\sigma(i)} = \sum_j a_j e_{\sigma(j)} = \sum_{\sigma^{-1}(i)} a_{\sigma^{-1}(i)} e_{\sigma(\sigma^{-1}(i))} = \sum_{\sigma^{-1}(i)} a_{\sigma^{-1}(i)} e_{i} = (a_{\sigma^{-1}(1)}, \ldots , a_{\sigma^{-1}(n)}), 
\]
as desired.

\item $\rho$ maps the permutations to the appropriate permutation matrices, which are $n\times n$-matrices of rank $n$ whose rows consist of zeros except for a single non-zero element, which is 1. The vector $(1, \ldots, 1) \in k^n$ is an eigenvector of any permutation matrix with eigenvalue 1. Therefore, the subspace spanned by $(1, \ldots, 1)$ constitutes a $S_n$-stable subspace $E$ of $k^n$ on which $S_n$ acts trivially. Therefore $\rho |_E$, the restriction of $\rho$ onto $E$, is a subrepresentation isomorphic to the trivial representation. 
\item Let $W$ be the orthogonal complement of $E$, so that $E + W = k^n$. Define $\rho' \coloneqq \rho |_W$. Denote $(1, 1, \ldots, 1) \coloneqq 1_n$. $W$ contains precisely the vectors $v$ such that $v\cdot 1_n = 0$, that is, the column sum of all $v\in W$ is 0. The action of $S_n$ on $W$ simply exchanges coordinates, which does not affect the row sum. Therefore $W$ is also $S_n$-stable, so $\rho |_W$ is a subrepresentation. Moreover, $E \oplus W = k^n$ and $E$ has dimension 1 (it has a basis $1_n$), so $W$ has dimension $n-1$. Hence $\rho = \rho |_E \oplus \rho |_W \cong \triv \oplus \rho|_W$, where $\rho|_W$ has dimension $n-1$, as desired.

\item Suppose we can indeed write $\rho$ as the product of two non-trivial subspaces. By Theorem 1.10. in Isaacs, this is equivalent to every $S_n$-stable subspace having a $S_n$-stable complement. Consider the subspace $E$ as in \textbf{(c)}. $E$ is the span of $(1, 1)$, so any candidate $W$ must be a space which is the span of a basis vector $w \in W$ which is linearly independent from $(1, 1)$. That is, $W = \Span((a, b), a \neq b)$. The only choice of $(a, b)$ that makes $W$ $S_n$-stable is putting $b= -a$ (since $S_n$ switches the coordinates of $(a, b)$), but this is not possible since $a = -a$ in characteristic 2, so $(a, -a) \in E$. Hence $E$ has no $S_n$-stable complement, which in turn proves that $\rho$ cannot be written as a direct sum of two nontrivial subspaces. The problem is precisely that $a \neq -a$ in characteristic 0, but not necessarily in other characteristics.

\item Suppose $\rho = \triv \oplus \rho'$, and suppose $p$ does not divide $n$. Again, we have the subspace $E$ spanned by $1_n$. The question is whether we can find $n-1$ linearly independent vectors, all of which are linearly independent to $1_n$, whose span is $G$-stable.   

\marginnote{INTE KLAR. Maschke?} 
\end{enumerate}
\end{proof}

\newpage



\begin{problem}
Some representations of cyclic and dihedral groups.
\begin{enumerate}[font=\normalfont,label=\textbf{(\alph*)}]
\item Let $n \geq 2$. Describe $n$ distinct, one-dimensional representations of $\mathbb{Z}/n$ over $\mathbb{C}$ and explain why they are pairwise non-isomorphic. 

\item Let $D_{2n} \coloneqq \langle x, y \mid x^n = y^2 = 1, yxy^{-1} = x^n \rangle$ be the dihedral group of order $2n$. Show that setting 
\[
\rho(x) \coloneqq
\begin{bmatrix}
\cos(2\pi/n) & -\sin(2\pi/n) \\
\sin(2\pi/n) & \cos(2\pi/n)
\end{bmatrix}
, \quad
\rho(y) \coloneqq
\begin{bmatrix}
0 & 1 \\
1 & 0
\end{bmatrix}
,
\]
gives a unique, well-defined representation of $D_{2n}$ over $\mathbb{R}$. 

\item Show that $\rho$ is irreducible over $\C$. [A priori, this is at least as strong as showing that $\rho$ is irreducible over $\R$.]

\item Show that the restriction $\Res$ of $\rho$ to $\langle x \rangle$ is irreducible over $\R$ if and only if $n\geq 3$. 

\item By contrast, show that the same restriction $\Res$ is always reducible over $\C$.
\end{enumerate}
\end{problem}

\begin{proof}[Solution]
\hfill
\begin{enumerate}[font=\normalfont,label=\textbf{(\alph*)}, wide]
\item Since $\Z/n$ is cyclic, any representation is completely determined by its mapping of a generator of $\Z/n$, say 1. The proposed representations are
\[
\begin{aligned}
\rho_k &: \Z/n \to \C, \\
1 &\mapsto \exp(2\pi i k/n), k = 1, \ldots, n, \\
\rho_k(l) &= p_k(1)^l.
\end{aligned}
\]
Indeed, this is a representation because 
\[
\rho_k(a + b) = \exp(2\pi i k/n)^{a+b}= \exp(2\pi i k/n)^{a} \exp(2\pi i k/n)^{b} = \rho_k(a)\rho_k(b). 
\]
These representations are also non-isomorphic because their characters are not equal. This can be seen by considering the character of a generator: $\rho_k(1) = \exp(2\pi i k/n) \neq \exp(2\pi i l/n)$ for $l \neq k, 1 \leq, k, l \leq n$. 

\item We have to verify that $x, y$ and $x^n, $
\end{enumerate}
\end{proof}

\newpage



\begin{problem}
Let $\rho$ be as in in Problem 1 with $k = \C$. 
\begin{enumerate}[font=\normalfont,label=\textbf{(\alph*)}]

\item Describe explicitly the characted $\chi_\rho$ of $\rho$: Given a permutation $\sigma$, what is the value $\chi_\rho(\sigma)$ in terms of its cycle type?

\item Assume a finite group $G$ acts on a finite set $X$. Let $\Fix (g) \coloneqq \{x\in X\mid gx = x\}$; the fixed points of $g$. What is $\sum_{g\in G}\lvert \Fix{g}\rvert$?

\item Deduce that $[\chi_\rho, \mathbb{1}] = 1$. 
\item Recall that $G$ acts doubly-transitively on $X$ if the action of $G$ on $X \times X$ has precisely two orbits (necessarily the diagonal and the rest). If $G$ acts doubly-transitively on $X$ and $r$ is the corresponding linear representation of $G$, show that $[\chi_r, \mathbb{1}] = 1 $ and $[\chi_r, \chi_r] = 2$. [Hint: Consider also the character of the linear representation gotten from the action of $G$ on $X\times X$.]

\item Deduce that $[\chi_{\rho'}, \chi_{\rho'}] =1$. 
\end{enumerate}
\end{problem}

\begin{proof}[Solution]
\hfill
\begin{enumerate}[font=\normalfont,label=\textbf{(\alph*)}, wide]
\item
\end{enumerate}
\end{proof}

\newpage



\begin{problem}
Let $k = \C$. Let $\rho$ be the representation of the dihedral group from Problem 2. 

\begin{enumerate}[font=\normalfont,label=\textbf{(\alph*)}]
\item Compute the character of $\rho$. 

\item Let $\chi, \psi$ be one-dimensional characters of a finite group $G$. Show that 
\[
(\chi, \psi) = \delta_{\rho, \psi} \coloneqq 
\begin{cases}
1, \chi = \psi, 
0, \text{ otherwise}
\end{cases}
\]
[This is the first orthogonality relation in the special case of one-dimensional characters. You may not cite the general orthogonality relation or use its proof here.]
\item Let $n \geq 2$. Show that $\sum_{m=1}^n e^{2\pi im/n} = 0$ in two ways: Use \textbf{(b)} and use the polynomial $x^n-1$. 
\item Show that $[\chi_\rho, \chi_\rho] = 1$. [Hint: Use \textbf{(c)}. I found it helpful to distinguish between the cases $n$ even and $n$ odd.]
\end{enumerate}
\end{problem}

\begin{proof}[Solution]
\hfill
\begin{enumerate}[font=\normalfont,label=\textbf{(\alph*)}, wide]
\item
\end{enumerate}
\end{proof}

\newpage


\begin{problem}
Assume $k$ is algebraically closed and $\Char(k) \neq 2$. 

\begin{enumerate}[font=\normalfont,label=\textbf{(\alph*)}]
\item Let $(V, \rho)$ be a self-dual irreducible representation of $G$ over $k$. Show that $\rho$ is either symplectic or orthogonal.

\item Assume $(V, \rho)$ is a representation which is both symplectic and orthogonal. Show that $(V, \rho)$ is reducible. 

\end{enumerate}
\end{problem}

\begin{proof}[Solution]
\hfill
\begin{enumerate}[font=\normalfont,label=\textbf{(\alph*)}, wide]
\item
\end{enumerate}
\end{proof}

\newpage


\begin{problem}
Consider the two permutation representations 
\[
\rho_1, \rho_2 : \Z/2 \times \Z/2 \to S_6
\]
given by 
\[
\rho_1 ((1, 0)) = \rho_2((1, 0)) = (12)(34), 
\]
while
\[
\rho_1 ((0, 1)) = (13)(24), \rho_2((0, 1)) = (12)(56).
\]
\begin{enumerate}[font=\normalfont,label=\textbf{(\alph*)}]
\item Show that $\rho_1$ and $\rho_2$ are element-conjugate. 
\item Show that $\rho_1$ and $\rho_2$ are not globally conjugate.

\end{enumerate}
\end{problem}

\begin{proof}[Solution]
\hfill
\begin{enumerate}[font=\normalfont,label=\textbf{(\alph*)}, wide]
\item Of the given cases, we need only examine how $\rho_1 ((0, 1))$ relates to $\rho_2((0, 1))$. What we have to show is that there is some $\tau \in S_6$ such that $\tau \rho_2((0, 1)) \tau^{-1} = \rho_1((0, 1))$. Take $\tau = (235)(64)$. Then
\[
\tau \rho_2((0, 1)) \tau^{-1} = (235)(64)(12)(56)(64)(532) = (13)(24). 
\]
By representations being homomorphisms, we have that $\rho_1((1, 1)) = \rho_1((1, 0) + (0,1)) =  \rho_1((1, 0)) \rho_1((0, 1)) = (12)(34)(13)(24) = (14)(23)$ and $\rho_2((1, 1)) = \rho_2((1, 0) + (0,1)) =  \rho_2((1, 0)) \rho_2((0, 1)) = (12)(34)(12)(56) = (34)(56)$. take $\sigma = (31526)$. Then
\[
\sigma\rho_1((1, 1))\sigma^{-1} = (31526)(34)(56)(62513) = (14)(23) = \rho_2((1,1)).
\]
Hence we have proved that $\rho_1$ and $\rho_2$ are element-conjugate. 

\item Suppose they are globally conjugate by $\tau$. Then we must have that
\[
\begin{aligned}
\tau \rho_1((0, 1))\tau^{-1} = \rho_2((0, 1)) &\iff \tau (13)(24)\tau^{-1} = (12)(56), \\
\tau \rho_1((1, 1))\tau^{-1} = \rho_2((1, 1)) &\iff \tau (14)(23)\tau^{-1} = (34)(56). 
\end{aligned}
\]
Now, $(12)(56) = [(13)(24)](34)(56)[(13)(24)]$, so substituting in the above we get
\[
\begin{aligned}
\tau (13)(24)\tau^{-1} &= (13)(24)\tau (14)(23)\tau^{-1} (13)(24) \\
\iff \tau (13)(24)\tau^{-1} &= (13)(24)\tau (14)(23)\tau^{-1} (13)(24)\tau\tau^{-1}
\end{aligned}
\]
\end{enumerate}
\end{proof}
\end{document}