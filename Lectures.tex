\documentclass[11pt]{amsart}
\usepackage[marginratio=1:1, headheight=27pt,
footskip = 13pt, a4paper, total={6.5in, 8in}]{geometry}                % See geometry.pdf to learn the layout options. There are lots.
\geometry{letterpaper}                   % ... or a4paper or a5paper or ...
%\geometry{landscape}                % Activate for for rotated page geometry
%\usepackage[parfill]{parskip}    % Activate to begin paragraphs with an empty line rather than an indent
\usepackage{graphicx}
\usepackage[font=small,labelfont=bf]{caption} % Required for specifying captions to tables
\usepackage{dirtytalk}
%\usepackage[version=3]{mhchem}
%\usepackage{subcaption}
\usepackage{subfig}
%\usepackage{epstopdf}
\usepackage{booktabs}
\usepackage[compact,small]{titlesec}
\usepackage{complexity}
%\usepackage[sc, osf]{mathpazo} % add possibly `sc` and `osf` options
%\usepackage{eulervm}
%\usepackage{textcomp}

\clubpenalty = 10000
\widowpenalty = 10000
%\usepackage[altbullet]{lucidabr}
\usepackage{float}
\usepackage{siunitx}
\usepackage[justification=centering]{caption}
\usepackage[
colorlinks = true,
urlcolor = blue
]{hyperref}
\usepackage{mathrsfs}
\usepackage{amssymb}
\usepackage{mathtools}
\usepackage{amsmath}
\usepackage{amsthm}
\renewcommand\qedsymbol{$\blacksquare$}

\usepackage[section]{placeins}
\DeclareGraphicsRule{.tif}{png}{.png}{`convert #1 `dirname #1`/`basename #1 .tif`.png}

%\usepackage[swedish]{babel}
\usepackage[T1]{fontenc}
\usepackage[utf8]{inputenc}

\usepackage{comment}
\usepackage{enumitem}
\titleformat{\section}[block]{\Large\scshape\centering}{\thesection.}{1em}{}
\titleformat{\subsection}[block]{\Large}{\thesubsection.}{1em}{}

\usepackage{pgfplots}
\pgfplotsset{compat = 1.15}

\usepackage[T1]{fontenc}

\usepackage{listings}
\lstset{basicstyle=\ttfamily,breaklines=true}
\usepackage{color} %red, green, blue, yellow, cyan, magenta, black, white
\definecolor{mygreen}{RGB}{28,172,0} % color values Red, Green, Blue
\definecolor{mylilas}{RGB}{170,55,241}

%\usepackage{times}
%\usepackage{kpfonts}
%\usepackage{txfonts}
%\usepackage{newtx}
%\usepackage{stix}
%\usepackage[osf,proportional]{libertine}
%\usepackage{lmodern}
\usepackage[charter]{mathdesign}

\usepackage{microtype}
%\usepackage{stix}
\usepackage[compact,small]{titlesec}
\titleformat*{\section}{\Large\bfseries\scshape}

\titleformat*{\subsection}{\large\bfseries\scshape}
\titleformat*{\subsubsection}{\large\bfseries\scshape}
%\renewcommand{\section}{\section*}
\newcommand{\ibits}{\{0, 1\}^*}
%\renewcommand{\familydefault}{\sfdefault}
\usepackage{marginnote}
\renewcommand*{\marginfont}{\color{red}\sffamily\footnotesize}
\usepackage{hyperref, xcolor}

%\usepackage{makeidx}
\definecolor{winered}{rgb}{0.5,0,0}
\hypersetup{
     pdfauthor={JAG},
     pdfsubject={Hyperlinks in LaTeX},
     pdftitle={main.tex},
     pdfkeywords={LaTeX, PDF, hyperlinks}
%    colorlinks=false,
     pdfborder={0 0 0},
%You can set individual colors for links as below:
colorlinks=true,
  linkcolor=winered,
urlcolor={winered},
filecolor={winered},
citecolor={winered},
allcolors={winered}
}

\usepackage[english]{babel} 
\usepackage[
backend=biber,
style=numeric,
hyperref=true,
%natbib
]{biblatex}
\DeclareLanguageMapping{swedish}{swedish-apa}
\addbibresource{Komplexitetsteori.bib}

\usepackage[T1]{fontenc}

%\usepackage{fouriernc}
\usepackage{tikz-cd}
\usepackage{thmtools}
\usepackage{fancyhdr}
\pagestyle{plain}
\usepackage{csquotes}
\newsavebox{\myheadbox}
\fancyhf{}
%\begin{flushright}
\lhead{
Jonas Conneryd \\ \url{conneryd@kth.se}}
%\end{flushright}
\rhead{970731-7559  \\
\the\year
}
\chead{\Large{\textbf{\scshape{Lectures } \\ \vspace{-4pt}\scshape{\normalsize MM8021 Representation Theory for Finite Groups}}}}
\cfoot{\thepage}
\declaretheoremstyle[headfont=\bfseries\sffamily]{normalhead}
\interfootnotelinepenalty=10000
%\title{\vspace{-2cm}\textbf{\textsf{ Homework 3}}}
\date{}
%\author{Jonas Conneryd \\
%conneryd@kth.se \\ 970731-7559}

\renewcommand{\C}{\mathbb{C}}
\newcommand{\GL}{\mathsf{GL}}
\newcommand{\Z}{\mathbb{Z}}
\renewcommand{\R}{\mathbb{R}}
\newcommand{\Res}{\mathsf{Res}^{D_{2n}}_{\langle x\rangle}}
\newcommand{\triv}{\mathbb{1}}
\newcommand{\Char}{\mathsf{Char}}
\newcommand{\Fix}{\mathsf{Fix}}
\newcommand{\Span}{\mathsf{Span}}
\newcommand{\Tr}{\mathsf{Tr}}
\newcommand{\Aut}{\mathsf{Aut}}
\let\emph\relax % there's no \RedeclareTextFontCommand
\DeclareTextFontCommand{\emph}{\bfseries\em}


\begin{document}
%\maketitle
\thispagestyle{fancy}
\theoremstyle{normalhead}
\newtheorem{theorem}{Theorem}[section]
\newtheorem{lemma}{Lemma}[section]

\theoremstyle{normalhead}
\newtheorem{definition}{Definition}[section]
\theoremstyle{remark}
\newtheorem*{remark}{Remark}
\theoremstyle{remark}
\newtheorem*{example}{Example}

\section{Lecture 1}

\subsection{The Group Ring}
\begin{definition}
Let $G$ be a group, $R$ a commutative ring with unity. The \emph{group ring} $R[G]$ is the set of formal linear combinations $\sum_{g\in G}a_g g$ with $a_g \in R$, such that $a_g = 0$ for all but finitely many $g\in G, a_g\in R$. We can turn $R[G]$ into a ring (commutative iff $G$ abelian) by defining addition componentwise:
\[
\sum a_g g + \sum b_g g = \sum (a_g + b_g)g, 
\]
as well as multiplying elements $g\in G$ using the group operation in $G$ and extend linearly to $R[G]$.
\end{definition}
\begin{example}
If $R = F$ is a field, then $F[G]$ is an $F$-vector space with basis $G$. 
\end{example}
\subsection{Representations}
Informally, a representation has 4 components: A field $F$, a group $G$, an $F$-vector space $V$, and a group homomorphism $G \overset{\rho}{\to} \Aut(V)$. When $V$ is finite-dimensional, $\Aut(V) = \GL(V)$ and for a basis $a_1, \ldots, a_n$ of $V$ we have that $\GL(V) \cong \GL(n)F$.
\begin{definition}
A \emph{representation} of $G$ on $V$ is a homomorphism $G \overset{\rho}{\to} \GL(V)$; equivalently, it is a linear group action of $G$ on $V$. 
\end{definition}
\begin{remark}
The action of $G$ on a set $X$ induces a homomorphism $\rho: G \to S_X$. For a finite set $X$ of size $n$, $S_x \cong S_n$. When $G$ acts on a \textit{vector space} $V$, we want a \textit{linear} action, so we get an induced homomorphism into the automorphism group of $V$.
\end{remark}
\begin{definition}
A \emph{subrepresentation} of $(V, \rho)$ is a vector subspace $W \subseteq V$ which is \emph{$G$-stable}: $g \cdot w \in W \forall g \in G$. 
\end{definition}
\begin{definition}
A representation is \emph{irreducible} or \emph{simple} if its only $G$-stable subspaces are $0$ and $V$. 
\end{definition}
\begin{definition}
A representation is \emph{faithful} (injective) if $\rho: G \to \GL(V)$ is injective.
\end{definition}

A question that arises is, given $V$ and subspaces $W \in V$, can we find a $G$-stable complement $W'$ such that $W \oplus W'= V$? 

\subsection{Some operations on a representation}
\begin{definition}
The \emph{direct sum} of two representations $\rho_1: G \to \GL(V_1), \rho_2: G \to \GL(V_2)$ is 
\[
\rho_1 \oplus \rho_2: G \to V_1 \oplus V_2; (\rho_1 \oplus \rho_2)(g) = (\rho_1(g), \rho_2(g)).
\]
\end{definition}
\begin{remark}
If $\GL(V_1) = (n, F), \GL(V_2) = (m, F)$ then 
\[
\rho_1(g) = A; \rho_2(g) = B; (\rho_1 \oplus \rho_2)(g) = 
\begin{bmatrix}
A & 0 \\
0 & B
\end{bmatrix}
.
\]
\end{remark}
\subsection{Semisimple representations}
\begin{theorem}
TFAE:
\begin{enumerate}
    \item $V$ is a direct sum of irreps.
    \item $V$ is a sum of irreps.
    \item Every $G$-stable subspace has a $G$-stable complement.
\end{enumerate}
\end{theorem}
\begin{definition}
If 1.-3. hold, $V$ is \emph{semisimple}. 
\end{definition}
\begin{theorem}[Maschke]
If $F = \C$ or $F$ is algebraically closed and $\Char(F) \not | \lvert G \rvert$, then every representation of the finite group $G$ is semistable.
\end{theorem}







\section{Lecture 2}
\subsection{Morphisms of representations}
\begin{definition}
A \emph{morphism} between representations $(V, \rho), (W, \sigma)$ is a linear map $\phi: V \to W$ such that the diagram
\begin{center}
\begin{tikzcd}
\arrow[swap,d, "\rho(g)"] V  \arrow[r, "\phi"] & W\arrow[d, "\sigma(g)"] \\
V \arrow[swap,r, "\phi"] & W
\end{tikzcd}
\end{center}
commutes. We say that the two representations are \emph{equivariant} or \emph{intertwining}. 
\end{definition}
\begin{definition}
Suppose $\rho, \sigma: G \to H$ are group homomorphisms (we are interested in the case of $H = \GL(V)$). We say that $\rho, \sigma$ are \emph{globally conjugate} if $\exists h \in H: \forall g\in G, h\rho(g)h^{-1} = \sigma(g)$. We say that $\rho, \sigma$ are \emph{locally conjugate} if $\forall g \in G \exists h \in H: h_g\rho(g)h_g^{-1} = \sigma(g)$. 
\end{definition}
\subsection{Characters}

\begin{definition}
Given $\rho, G \to \GL(V), V/F$, the \emph{character} of $\rho$ is the function
\[
\chi_\rho : G \to F; g \mapsto \Tr(\rho(g)).
\]
\end{definition}
\begin{remark}
$\rho \cong \sigma \implies \chi_\rho = \chi_\sigma$ since $\Tr(AB) = \Tr(BA)$. 
\end{remark}
A natural question to ask is if the converse of the above holds. It turns out that for finite groups and $F = \C$, the answer is \emph{yes}!

\end{document}