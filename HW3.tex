\input{preamble.tex}
\usepackage[T1]{fontenc}

%\usepackage{fouriernc}

\usepackage{thmtools}
\usepackage{fancyhdr}

\usepackage{csquotes}

\usepackage{polynom}
\polyset{%
   style=C,
   delims={\big(}{\big)},
   div=:
}

\newsavebox{\myheadbox}
\fancypagestyle{normalpage}
{
%\begin{flushright}
\lhead{
Jonas Conneryd
}
%\end{flushright}
\rhead{\url{conneryd@kth.se}}
\chead{MM8021 Representation Theory of Finite Groups}
\cfoot{\thepage}
}
\fancyhf{}
\fancypagestyle{firstpage}
{
%\begin{flushright}
\lhead{
Jonas Conneryd \\ \url{conneryd@kth.se}}
%\end{flushright}
\rhead{970731-7559  \\
\the\year
}
\chead{\Large{\scshape{\textbf{Homework 2} \\ \vspace{-4pt}\normalsize\textbf{ MM8021 Representation Theory of Finite Groups}}}}
\cfoot{\thepage}
}

\pagestyle{normalpage}


\declaretheoremstyle[headfont=\bfseries\scshape]{normalhead}
\interfootnotelinepenalty=10000
%\title{\vspace{-2cm}\textbf{\textsf{ Homework 3}}}
\date{}
%\author{Jonas Conneryd \\
%conneryd@kth.se \\ 970731-7559}

\renewcommand{\C}{\mathbb{C}}
\newcommand{\GL}{\mathsf{GL}}
\newcommand{\Z}{\mathbb{Z}}
\renewcommand{\R}{\mathbb{R}}
\newcommand{\Res}{\mathsf{Res}^{D_{2n}}_{\langle x\rangle}}
\newcommand{\triv}{\mathbb{1}}
\newcommand{\Char}{\mathsf{Char}}
\newcommand{\Fix}{\mathsf{Fix}}
\newcommand{\Span}{\mathsf{Span}}
\newcommand{\Tr}{\mathsf{Tr}}
\newcommand{\Ker}{\mathsf{Ker}}
\renewcommand{\Im}{\mathsf{Im}}
\newcommand{\Cent}{\mathsf{Cent}}
\newcommand{\Irr}{\mathsf{Irr}}

\begin{document}
%\maketitle
\thispagestyle{firstpage}
\theoremstyle{normalhead}
\newtheorem{problem}{Problem}
\newtheorem{lemma}{Lemma}

\noindent {\large \textbf{I wish I had more time to spend on this assignment. As a result of this lack of time, some solutions are incomplete, and others are nonsensical. I apologize in advance for this.}}

\begin{problem}[2.3]
Let $\chi$ be a character of $G$. Define $\det {\chi}: G \to \C$ as follows: Choose $\rho$ affording $\chi$ and set
\[
(\det{\chi})(g) = \det{\rho(g)}.
\]
Show that $\det{\chi}$ is a uniquely defined linear character of $G$.

\end{problem}

\begin{proof}[{\scshape Solution}]
To prove uniqueness, we must prove that $\det{\chi}$ does not depend on the choice of $\rho$. Let $\rho$ and $\psi$ be representations affording $\chi$. Since $\rho$ and $\psi$ both afford $\chi$, we know that they are similar by Corollary 2.9: $\rho(g) = P\psi(g) P^{-1}$ for some matrix $P$. Then by the multiplicativity of $\det$ we have
$\det(\rho(g)) = \det(P\psi(g)P^{-1}) = \det(P)\det(\psi(g))\det(P^{-1}) = \det(P)\det(P^{-1})\det(\psi(g)) = \det(P P^{-1})\det(\psi(g)) = \det(\psi(g))$.

Now for the degree of $\det(\chi)$. Any representation $\rho$ has $\rho(1) = \triv$, so $\det{\chi(1)} = \det(\triv) = 1$, so $\det{\chi}$ is linear. Therefore it is a homomorphism from $G$ to $\C$ and thus a character of degree 1.


\end{proof}

\newpage

\begin{problem}[2.4]
  \hfill
\begin{enumerate}[font=\normalfont,label=\textbf{(\Alph*)}]


  \item Let $G$ be a nonabelian group of order 8. Show that $G$ has a unique nonlinear irreducible character $\chi$. Show that $\chi(1) = 2, \chi(z) = -2, \chi(x) = 0$, where $z \in G'-\{1\}$ and $x\in G-G'$.
  \item If $G \cong D_8$, show that $\det{\chi} \neq 1_G$.
  \item If $G \cong Q_8$, show that $\det{\chi} = 1_G$.
\end{enumerate}
\textbf{HINT: Show that $\Ker(\det{\chi})$ contains all elements of order 4. Use Lemma 2.15.} DIHEDRAL HAR 2 element av order 4, Sym har 6
\end{problem}

\begin{proof}[{\scshape Solution}]
  \hfill
  \begin{enumerate}[font=\normalfont,label=\textbf{(\Alph*)}, wide]


    \item Suppose $G$ is nonabelian of order 8. We know by Corollary 2.6. that a group is abelian iff all its irreducible characters are linear, so we know that there is at least one nonlinear irreducible character. Let $k$ be the number of conjugacy classes of $G$, which we know is not 8. Then
    \[
    8 = \sum_{i=1}^k \chi_i(1)^2,
    \]
    where we know that the $\chi_i(1)$ are integers. There are two possibilities: either there are two conjugacy classes $1, 2$ and both of their characters $\chi_1(1)$ and $\chi_2(1)$ fulfill $ \chi_i(1) =2$, or there is one conjugacy class $i$ with $\chi_i(1)= 2$ and four conjugacy classes who have linear characters. But certainly the trivial character is linear, so the first option is not possible. We conclude that $G$ has a unique nonlinear irreducible character $\chi$ and that $\chi(1) = 2$.

    By Corollary 2.23 we know that $\lvert G : G' \rvert = \text{ the number of linear characters in $G$}$, and by the above paragraph we know that this number is 4. Therefore Lagrange's theorem implies that there are two elements in $G'$, one of which is the identity and one of which we denote by $z$. Certainly no element except for the identity is conjugate to the identity.
\begin{comment}
 It is clear that this implies that $z = z^{-1}$.
    Since the commutator subgroup is normal, $gG'g^{-1} = G'$ for all $g\in G$. However we cannot have $gzg^{-1} = 1$ since $z$ is not the identity, so it follows that $gzg^{-1} = z$ for all $g \in G$.
\end{comment}
    Also, by Corollary 2.23, $\chi(z) = 1$ on linear characters. By the second orthogonality relation
    \[
    \begin{aligned}
      0
      &=  \sum_{\chi' \in \Irr(G)} \chi'(1) \overline{\chi'(z)} \\
      &= \sum_{\chi' \text{ linear } \in \Irr(G)} \chi'(1) \overline{\chi'(z)} + \chi(1)\overline{\chi(z)}\\
      &= 4 + 2\overline{\chi(z)}
    \end{aligned}
    \]
    so $\chi(z) = \overline{\chi(z)} = -2$. Finally, by the first orthogonality relation
    \[
    \begin{aligned}
      8 = \lvert G \rvert
      &= \sum_{g \in G} \chi(g)\chi(g^{-1}) \\
      &= \sum_{g \in G} \chi(g)\overline{\chi(g)} \\
      &= \sum_{g \in G'} \chi(g)\overline{\chi(g)} + \sum_{g \in G-G'} \chi(g)\overline{\chi(g)} \\
      &= 2^2 + (-2)^2 + \sum_{g \in G-G'} \chi(g)\overline{\chi(g)} \\
      &= 8 + \sum_{g \in G-G'} \chi(g)\overline{\chi(g)},
    \end{aligned}
    \]
    which implies that $\chi(g) = 0$ for all $g \in G-G'$. This completes the proof.

    \item Note that $D_8 = \langle s, r \mid s^2, r^8, srsr \rangle$ has two elements of order 4, namely $r$ and $sr$. Note that $\det{\chi}$ is a linear character. Therefore, for an element $d$ of order 4 we have $1 = \det{\chi}(1) = \det{\chi}(d^4) = \det{\chi}(d)^4$. At the same time, we can write $\chi(d) = \sum_i \varepsilon_i = 0$ and $\det{\chi}(d) = \Pi_i \varepsilon_i$. Therefore we can conclude that $\Pi_i \varepsilon_i$ has modulus 1, which by Lemma 2.15. means that each of the $\varepsilon_i$ have modulus 1. At the same time, their sum is 0. Finally, the $\varepsilon_i$ are all of order 4, so $\chi()$

    The elements of order 4 are all in $G - G'$.
    \item Recall that $Q_8$ has presentation
    $\langle \bar{e}, i, j, k \mid \bar{e}^2 = e, i^2 = j^2 = k^2 = ijk = \bar{e} \rangle$ where $\bar{e}$ commutes with the other elements of the group. Suppose we know that all elements of order 4 are in the kernel of $\det{\chi}$. From the relation $i^2 = j^2 = k^2 = ijk = \bar{e}$ we see that we can generate all elements of $Q_8$ from two elements of order four (these are $i, j, k$). Therefore, by linearity of the determinant, it follows that if all elements of order 4 are in the kernel of the determinant $\det{\chi}$, then $\det{\chi}$ is the identity map.

    Note that $\det{\chi}(ijk) = \det{\chi}(i^2) = \det{\chi}(j^2) = \det{\chi}(k^2) = \det{\chi}(\bar{e})$. By linearity all of these must square to 1. But then
  \end{enumerate}
\end{proof}

\newpage
\begin{problem}[2.5]
\hfill

\begin{enumerate}[font=\normalfont,label=\textbf{(\Alph*)}]
  \item Find a real representation of $D_8$ which affords the character $\chi$ of problem 2.4 \textbf{(A)}.
  \item Show that this cannot be done for the group $Q_8$.
\end{enumerate}
\end{problem}

\begin{proof}[{\scshape Solution}]
  \hfill
  \begin{enumerate}[font=\normalfont,label=\textbf{(\Alph*)}, wide]
    \item
    Recall that the dihedral group $D_8$ has presentation $\langle r, s | s^2, r^4 \rangle$
    Consider the representation
    \[
      \begin{aligned}
        r \mapsto
        \begin{bmatrix}
          0 & -1 \\
          1 &  0
        \end{bmatrix}
        \qquad
        s \mapsto
        \begin{bmatrix}
          1 & 0 \\
          0 & -1
        \end{bmatrix}
        \\
      \end{aligned}
    \]
    It is not hard to see that $G' = \{1, s\}$ and $G-G' = \{r^i, sr^i\}, i < 4$. Seeing that the required trace formulae are satisfied is now a matter of computation, which we omit from this document.

    \item Recall that the $\det(\chi)$ of such a representation is the identity representation.
  \end{enumerate}
\end{proof}

\newpage
\begin{problem}[2.8]
Let $\chi$ be a faithful character of $G$. Show that $H \subseteq G$ is abelian iff every irreducible constituent of $\chi_H$ is linear.
\end{problem}

\begin{proof}[{\scshape Solution}]
Recall that a group is abelian iff all its irreducible characters are linear, by Corollary 2.6. Suppose $H \subseteq G$ is abelian. Then every irreducible constituent of $\chi_H$ is linear in light of the fact stated above, since they are characters of $H$.

Conversely, suppose every irreducible constituent of $\chi_H$ is linear. Since $\chi$ is faithful over $G$, we get that $\chi_H$ is faithful over $H$. Consider the commutator subgroup $H'$ of $H$. We know that $G' = \cap\{\Ker(\lambda) : \lambda \in \Irr(H), \lambda(1) = 1\}$. Therefore, the commutator subgroup of $H$ is contained in the intersection of the kernels of the (linear) irreducible constituents of $\chi_H$. But the intersection of the kernels of the irreducible constituents of $\chi_H$ is in the kernel of $\chi_H$, and since $\chi_H$ is faithful, $\Ker(\chi_H) = 1$. Therefore the commutator subgroup of $H$ is trivial, so $H$ is abelian.
\end{proof}

\newpage
\begin{problem}[2.9]
\hfill
\begin{enumerate}[font=\normalfont,label=\textbf{(\Alph*)}]
  \item Let $\chi$ be a character of an abelian group $A$. Show that
  \[
    \sum_{a \in A} \lvert \chi(x)\rvert^2 \geq \lvert A \rvert \chi(1).
  \]
  \item Let $A\subseteq G$ with $A$ abelian and $\lvert G : A \rvert = n$. Show that $\chi(1) \leq n$ for all $\chi \in \Irr(G)$.
\end{enumerate}
\end{problem}

\begin{proof}[{\scshape Solution}]
  \hfill
  \begin{enumerate}[font=\normalfont,label=\textbf{(\Alph*)}, wide]
    \item By Corollary 2.17, we have that $[\chi, \chi] \geq 1$. Therefore
    \[
    \begin{aligned}
      [\chi, \chi] &= \frac{1}{\lvert A \rvert}\sum_{a \in A} \chi(a)\overline{\chi(a)} = \frac{1}{\lvert A \rvert}\sum_{a \in A} \lvert \chi(a) \rvert^2 \geq 1 \\
      \iff &\sum_{a \in A} \lvert \chi(a) \rvert^2 \geq \lvert A \rvert.
    \end{aligned}
    \]
    In an abelian group all characters are linear, so $\chi(1) = 1$. Therefore
    \[
      \sum_{a \in A} \lvert \chi(a) \rvert^2 \geq \lvert A \rvert = \lvert A \rvert \chi(1).
    \]
    The proof is complete.
    \item Let $\chi \in \Irr(G)$. By part \textbf{(A)} and the first orthogonality relation we have
    \[
      \lvert G \rvert = \sum_{g \in G} \lvert \chi(g) \rvert^2 \geq \sum_{a \in A} \lvert \chi(a) \rvert^2 \geq \lvert A \rvert = \lvert A \rvert \chi_H(1).
    \]
    Since $A$ is abelian, $A = Z(A)$. By Corollary 2.28,
\marginnote{EJ KLAR}
  \end{enumerate}
\end{proof}

\newpage
\begin{problem}[2.10]
Suppose $G = \bigcup_{i=1}^n A_i$, where the $A_i$ are abelian subgroups of $G$ and $A_i \cap A_j = 1$ if $i \neq j$.

\begin{enumerate}[font=\normalfont,label=\textbf{(\Alph*)}]
  \item Let $\chi \in \Irr(G)$. Show that if $\chi(1) > 1$, then $\chi(1) \geq \lvert G \rvert /(n-1)$.
  \item If $G$ is nonabelian, then $\lvert A_i \rvert \leq n-1$ for each $i$ and $n-1 \geq \lvert G \rvert^{1/2}$.
\end{enumerate}
\end{problem}

\begin{proof}[{\scshape Solution}]
  \hfill
  \begin{enumerate}[font=\normalfont,label=\textbf{(\Alph*)}, wide]
    \item
    \begin{comment}

    Since all the subgroups are disjoint, we have that $\lvert G \rvert = \sum_i^n \lvert A_i \rvert - (n-1)$. By problem 2.9. we know that $\chi(1) \leq \lvert G : A_i \rvert$ for all $A_i$. If $\chi(1) > 1$, we know that at least one of the $A_i$ has $\lvert A_i \rvert > 1$. Let $A_j$ be of maximal size among the $A_i$. Then $\lvert G \rvert /(n-1) =  \sum_i^n \lvert A_i \rvert/\lvert A_i \rvert - n-1$.


    Since the subgroups $A_i$ are abelian, we know that $\chi_{A_i}(1) = 1$. Since the $A_i$ are disjoint and together make up $G$, we have that $\chi(1),  = \sum_{A_i}\chi_{A_i}(1) - n-1(\chi(1)) = n-(n-1)$. Now


    By lemma 2.29 we kn

    By the second orthogonality relation
    \[
      \lvert G \rvert = \sum_{\Irr(G)}\chi(1)\overline{\chi(1)}.
    \]
    The conjugacy classes are precisely the ingoing abelian groups in the union, so
\end{comment}
    \item If $G$ is nonabelian, there is at least one irreducible nonlinear character $\chi$. By part \textbf{(A)}, we have that $\chi(1) \geq \lvert G \rvert /(n-1)$. Moreover, by Problem 2.9. we know that $\chi(1) \leq \lvert G : A_i \rvert = \lvert G \rvert / \lvert A_i \rvert $. Therefore $\lvert G \rvert / \lvert A_i \rvert \geq \chi(1) \geq \lvert G \rvert /(n-1)$, so $\lvert A_i \rvert \leq n-1$. It follows immediately that $n-1 \geq \lvert G \rvert /\chi(1)$. By Corollary 2.30 we also have $\chi(1)^2 \leq \lvert G : Z(\chi)\rvert \leq \lvert G \rvert$ for $\chi(g) \in \Irr(G)$, from which it follows that $\chi(1) \leq \lvert G \rvert^{1/2}$. Hence $n-1 \geq \lvert G \rvert /\chi(1) \geq \lvert G \rvert / \lvert G \rvert^{1/2} = \lvert G \rvert^{1/2}$, which is our desired result.

  \end{enumerate}
\end{proof}

\newpage
\begin{problem}[2.11]
Let $g\in G$. Show that $g$ is conjugate to $g^{-1}$ in $G$ iff $\chi(g)$ is real for all characters $\chi$ of $G$.
\end{problem}

\begin{proof}[{\scshape Solution}]
Suppose $\chi(g)$ is real for all characters $\chi$ of $G$. In any case, for $\chi = \triv$ we have $\triv(g)\triv(g^{-1}) = \triv(g)\overline{\triv(g)} = 1$. Then we have
\[
\sum_{\chi \in \Irr(g)}\chi(g)\overline{\chi(g^{-1})} = \sum_{\chi \in \Irr(g)}\chi(g)^2 > 0,
\]
since $\chi(g)$ are real and not all zero because the trivial character is nonzero. By the second orthogonality relation, $g$ and $g^{-1}$ are conjugate in $G$.

Conversely, suppose $g$ and $g^{-1}$ are conjugate in $G$. Then by Theorem 2.2.1. in the course notes, $\chi(g) = \chi(g^{-1})$ for all irreducible characters (and therefore all characters) $\chi$. On the other hand $\chi(g^{-1}) = \overline{\chi(g)}$ for all characters $\chi$, so $\overline{\chi(g)} = \chi(g)$ for all $\chi$. We conclude that $\chi(g)$ is real for all characters. The proof is complete.
\end{proof}

\end{document}
