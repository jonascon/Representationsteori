\documentclass[11pt]{amsart}
\usepackage[marginratio=1:1, headheight=27pt,
footskip = 13pt, a4paper, total={6.5in, 8in}]{geometry}                % See geometry.pdf to learn the layout options. There are lots.
\geometry{letterpaper}                   % ... or a4paper or a5paper or ...
%\geometry{landscape}                % Activate for for rotated page geometry
%\usepackage[parfill]{parskip}    % Activate to begin paragraphs with an empty line rather than an indent
\usepackage{graphicx}
\usepackage[font=small,labelfont=bf]{caption} % Required for specifying captions to tables
\usepackage{dirtytalk}
%\usepackage[version=3]{mhchem}
%\usepackage{subcaption}
\usepackage{subfig}
%\usepackage{epstopdf}
\usepackage{booktabs}
\usepackage[compact,small]{titlesec}
\usepackage{complexity}
%\usepackage[sc, osf]{mathpazo} % add possibly `sc` and `osf` options
%\usepackage{eulervm}
%\usepackage{textcomp}

\clubpenalty = 10000
\widowpenalty = 10000
%\usepackage[altbullet]{lucidabr}
\usepackage{float}
\usepackage{siunitx}
\usepackage[justification=centering]{caption}
\usepackage[
colorlinks = true,
urlcolor = blue
]{hyperref}
\usepackage{mathrsfs}
\usepackage{amssymb}
\usepackage{mathtools}
\usepackage{amsmath}
\usepackage{amsthm}
\renewcommand\qedsymbol{$\blacksquare$}

\usepackage[section]{placeins}
\DeclareGraphicsRule{.tif}{png}{.png}{`convert #1 `dirname #1`/`basename #1 .tif`.png}

%\usepackage[swedish]{babel}
\usepackage[T1]{fontenc}
\usepackage[utf8]{inputenc}

\usepackage{comment}
\usepackage{enumitem}
\titleformat{\section}[block]{\Large\scshape\centering}{\thesection.}{1em}{}
\titleformat{\subsection}[block]{\Large}{\thesubsection.}{1em}{}

\usepackage{pgfplots}
\pgfplotsset{compat = 1.15}

\usepackage[T1]{fontenc}

\usepackage{listings}
\lstset{basicstyle=\ttfamily,breaklines=true}
\usepackage{color} %red, green, blue, yellow, cyan, magenta, black, white
\definecolor{mygreen}{RGB}{28,172,0} % color values Red, Green, Blue
\definecolor{mylilas}{RGB}{170,55,241}

%\usepackage{times}
%\usepackage{kpfonts}
%\usepackage{txfonts}
%\usepackage{newtx}
%\usepackage{stix}
%\usepackage[osf,proportional]{libertine}
%\usepackage{lmodern}
\usepackage[charter]{mathdesign}

\usepackage{microtype}
%\usepackage{stix}
\usepackage[compact,small]{titlesec}
\titleformat*{\section}{\Large\bfseries\scshape}

\titleformat*{\subsection}{\large\bfseries\scshape}
\titleformat*{\subsubsection}{\large\bfseries\scshape}
%\renewcommand{\section}{\section*}
\newcommand{\ibits}{\{0, 1\}^*}
%\renewcommand{\familydefault}{\sfdefault}
\usepackage{marginnote}
\renewcommand*{\marginfont}{\color{red}\sffamily\footnotesize}
\usepackage{hyperref, xcolor}

%\usepackage{makeidx}
\definecolor{winered}{rgb}{0.5,0,0}
\hypersetup{
     pdfauthor={JAG},
     pdfsubject={Hyperlinks in LaTeX},
     pdftitle={main.tex},
     pdfkeywords={LaTeX, PDF, hyperlinks}
%    colorlinks=false,
     pdfborder={0 0 0},
%You can set individual colors for links as below:
colorlinks=true,
  linkcolor=winered,
urlcolor={winered},
filecolor={winered},
citecolor={winered},
allcolors={winered}
}

\usepackage[english]{babel} 
\usepackage[
backend=biber,
style=numeric,
hyperref=true,
%natbib
]{biblatex}
\DeclareLanguageMapping{swedish}{swedish-apa}
\addbibresource{Komplexitetsteori.bib}

\usepackage[T1]{fontenc}

%\usepackage{fouriernc}

\usepackage{thmtools}
\usepackage{fancyhdr}

\usepackage{csquotes}

\usepackage{polynom}
\polyset{%
   style=C,
   delims={\big(}{\big)},
   div=:
}

\newsavebox{\myheadbox}
\fancypagestyle{normalpage}
{
%\begin{flushright}
\lhead{
Jonas Conneryd
}
%\end{flushright}
\rhead{\url{conneryd@kth.se}}
\chead{MM8021 Representation Theory of Finite Groups}
\cfoot{\thepage}
}
\fancyhf{}
\fancypagestyle{firstpage}
{
%\begin{flushright}
\lhead{
Jonas Conneryd \\ \url{conneryd@kth.se}}
%\end{flushright}
\rhead{970731-7559  \\
\the\year
}
\chead{\Large{\scshape{\textbf{Homework 2} \\ \vspace{-4pt}\normalsize\textbf{ MM8021 Representation Theory of Finite Groups}}}}
\cfoot{\thepage}
}

\pagestyle{normalpage}


\declaretheoremstyle[headfont=\bfseries\scshape]{normalhead}
\interfootnotelinepenalty=10000
%\title{\vspace{-2cm}\textbf{\textsf{ Homework 3}}}
\date{}
%\author{Jonas Conneryd \\
%conneryd@kth.se \\ 970731-7559}

\renewcommand{\C}{\mathbb{C}}
\newcommand{\GL}{\mathsf{GL}}
\newcommand{\Z}{\mathbb{Z}}
\renewcommand{\R}{\mathbb{R}}
\newcommand{\Res}{\mathsf{Res}^{D_{2n}}_{\langle x\rangle}}
\newcommand{\triv}{\mathbb{1}}
\newcommand{\Char}{\mathsf{Char}}
\newcommand{\Fix}{\mathsf{Fix}}
\newcommand{\Span}{\mathsf{Span}}
\newcommand{\Tr}{\mathsf{Tr}}
\newcommand{\Ker}{\mathsf{Ker}}
\renewcommand{\Im}{\mathsf{Im}}
\newcommand{\Cent}{\mathsf{Cent}}
\newcommand{\Irr}{\mathsf{Irr}}

\begin{document}
%\maketitle
\thispagestyle{firstpage}
\theoremstyle{normalhead}
\newtheorem{problem}{Problem}
\newtheorem{lemma}{Lemma}

\begin{problem}[2.3]
Let $\chi$ be a character of $G$. Define $\det {\chi}: G \to \C$ as follows: Choose $\rho$ affording $\chi$ and set
\[
(\det{\chi})(g) = \det{\rho(g)}.
\]
Show that $\det{\chi}$ is a uniquely defined linear character of $G$.

\end{problem}

\begin{proof}[Solution]

\end{proof}

\newpage

\begin{problem}[2.4]
  \hfill
\begin{enumerate}[font=\normalfont,label=\textbf{(\Alph*)}]


  \item Let $G$ be a nonabelian group of order 8. Show that $G$ has a unique nonlinear irreducible character $\chi$. Show that $\chi(1) = 2, \chi(z) = -2, \chi(x) = 0$, where $z = G-\{1\}$ and $x\in G-G'$.
  \item If $G \cong D_8$, show that $\det{\chi} \neq 1_G$.
  \item If $G \cong Q_8$, show that $\det{\chi} = 1_G$.
\end{enumerate}
\textbf{HINT: Show that $\Ker(\det{\chi})$ contains all elements of order 4. Use Lemma 2.15.}
\end{problem}

\begin{proof}[Solution]

\end{proof}

\newpage
\begin{problem}[2.5]
\hfill

\begin{enumerate}[font=\normalfont,label=\textbf{(\Alph*)}]
  \item Find a real representation of $D_8$ which affords the character $\chi$ of problem 2.4 \textbf{(A)}.
  \item Show that this cannot be done for the group $Q_8$.
\end{enumerate}
\end{problem}

\begin{proof}[Solution]

\end{proof}

\newpage
\begin{problem}[2.8]
Let $\chi$ be a faithful character of $G$. Show that $H \subseteq G$ is abelian iff every irreducible constituent of $\chi_H$ is linear.
\end{problem}

\begin{proof}[Solution]

\end{proof}

\newpage
\begin{problem}[2.9]

\begin{enumerate}[font=\normalfont,label=\textbf{(\Alph*)}]
  \item Let $\chi$ be a character of an abelin group $A$. Show that
  \[
    \sum_{a \in A} \lvert \chi(x)\rvert^2 \geq \lvert A \rvert \chi(1).
  \]
  \item Let $A\subseteq G$ with $A$ abelian and $\lvert G : A \rvert = n$. Show that $\chi(1) \leq n$ for all $\chi \in \Irr(G)$.
\end{enumerate}
\end{problem}

\begin{proof}[Solution]

\end{proof}

\newpage
\begin{problem}[2.10]
Suppose $G = \bigcup_{i=1}^n A_i$, where the $A_i$ are abelian subgroups of $G$ and $A_i \cap A_j = 1$ if $i \neq j$.

\begin{enumerate}[font=\normalfont,label=\textbf{(\Alph*)}]
  \item Let $\chi \in \Irr(G)$. Show that if $\chi(1) > 1$, then $\chi(1) \geq \lvert G \rvert /(n-1)$.
  \item If $G$ is nonabelian, then $\lvert A_i \rvert \leq n-1$ for each $i$ and $n-1 \geq \lvert G \rvert^{1/2}$.
\end{enumerate}
\end{problem}

\begin{proof}[Solution]

\end{proof}

\newpage
\begin{problem}[2.11]
Let $g\in G$. SHow that $g$ is conjugate to $g^{-1}$ in $G$ iff $\chi(g)$ is real for all characters $\chi$ of $G$.
\end{problem}

\begin{proof}[Solution]

\end{proof}

\end{document}
